《金瓶梅》刻画了几十个性格各异、形象立体、层次分明的人物,这些人物在作者兰陵笑笑生笔下如同有了生命,于字里行间活脱跳跃。

\section{西门庆}
西门庆是本书的主角,是众多角色的主线。他父母早丧,兄弟俱无,靠生药铺艰难创业发家,有了点钱就在县衙门里包揽诉讼,利用自己强大的智商、情商和运气,做成了清河县的千户,掌管了这里的刑事诉讼,积累了万贯家财。在这个过程里,他不仅和当地的商人、官员、大户交结并捆绑在一条船上,还通过翟千户的关系与中央的蔡京等人建立了利益关系,可谓上下通吃。值得注意的是,西门庆的家产刚开始并不算特别富有,李瓶儿把自己的财产送给了他,才让他事业真正红火起来。

在地方上,西门庆可谓飞扬跋扈。他利用自己的职权,贪赃枉法,在诉讼上可谓败坏到了家。对等自己的“上级”,他摇尾乞怜,从他们手里获得权力,并实心办事,讨好上级。对等同级,他尽力拉扰,“强强联合”,又是通婚又是参股,把自己的家产越做越大。对等下位者和平民百姓,他如同君主,丝毫不放在眼里。

作为一部描述“性”毫不顾忌的小说,西门庆与其他女人间的性事成了本书最引人注目的情节。西门庆对待性,如同对待权力一样,可谓毫不遮掩毫无保留。他几乎天天做夜夜做,不仅和自家的妾做,还常常勾引下人,甚至连男人也不放过。他如同饿狼般地勾引其他的良家妇女。而这些和他交欢的女人,不仅仅是看中了他的权力和财富,还看中了他有够也愿意给女人带来“快乐”。王婆所说的“潘驴邓小闲”,西门庆样样都占。吊诡的是,对于自己的正妻吴月娘,他却很少下手。他前妻死亡,只留下一个女儿,李瓶儿为他生了儿子但被潘金莲害死,只有等到死亡当天才由吴月娘生下儿子,但最终儿子出家,也没有继承他的家业。

西门庆对权力和性的追求,在我看来是一体的。如同那句著名的话,\emph{Everything in the world is about sex except sex. Sex is about power.}西门庆对性和权力这种赤裸裸的毫无羞耻的追求,源于他生命底色里的阳刚。《金瓶梅》里的男人,从贩夫走卒到高级官员,甚至是读圣贤书的人,无一不是在性上贪婪无厌,,但做事如此“坦荡”“自然”的,恐怕只有西门庆一个。

极具宿命论的是,正是这种对性的贪婪无厌害死了西门庆。胡僧药对于西门庆来说,是追求性快感的捷径和武器,也掏空了他的身体。西门庆死于和潘金莲的交欢,而后者是比他更疯狂的性瘾者,真是“一物降一物”。他一死,附着在他身上的如同蛆虫般男男女女成了反噬他的蚁群,他的家产也很快被人瓜分殆尽。

\section{潘金莲}

潘金莲是本书的“女一号”,和西门庆一起撑起了大半部书。从时间上来看,本书开始于《水浒传》里的武松的故事,结束于小西门庆出家,潘金莲就占了将近90回的时间跨度,戏份颇重,因此说她是本书绝对的“一号”也不为过。

潘金莲生于裁缝之家,排行第六,大名叫“六姐儿”,小时裹脚,因此小名称为“金莲”。她本来就很聪明伶俐,擅长女红和乐器。九岁时被母亲卖到王招宣府里做童养媳,王招宣死后又被母亲潘老太以30两银子的价钱卖给张大户。张大户被母亲看得严,但没有儿女,娶来了潘金莲和白玉莲做继承香火,但白天让她们学习乐器(潘金莲学琵琶,白玉莲学筝),晚上才作为“使女”生孩子。后来白玉莲死去,只剩下潘金莲,张大户与她“偷情”(自己的侍女怎么能叫偷情?可见母亲管得实在太严了)后身体状况变差,被母亲发现把潘金莲打了一顿,后来把她转卖给自家的租户武大郎。这武大郎本来也有妻子,还留下一个小女儿迎儿,但妻子早死,他生性懦弱,同时还会奉承张大户,在张大户眼里认为他“老实”,于是把金莲许配给他,并趁机与金莲偷情。武大郎长得奇丑无比,又身材矮小,潘金莲很不满意,天天喝骂他,并在外偷情养汉子。

武松出现以后,一表人才身材高大并且充满阳刚之气的他很快吸引了潘金莲的目光,潘金莲开始勾引他,但武松心里的道德感和伦理感丝毫不给潘金莲机会。武松因公事离开清河县,潘金莲继续勾搭人,并遇见了西门庆,本书的真正故事随即开始。

之后潘金莲与西门庆勾引并毒死武大郎的故事在《水浒传》里也有大致相同的情节,在中国可谓家喻户晓了。武大郎死后,西门庆瞒天过海把潘金莲收了房,称为“五娘”。在西门庆家,潘金莲的泼辣表现得毫一遮掩。她与西门庆交欢的频率应该是最高的,同时也开始勾引陈敬济,并在西门庆死后得手。李瓶儿来了以后,受到西门庆的更大的宠爱,引起了潘金莲的妒忌,这种心理在李瓶儿为西门庆生下儿子伴随着西门庆对李瓶儿更大的宠爱而变得更加浓烈。潘金莲设计害死了这个儿子官哥儿并间接害死了李瓶儿,利用自己在性事和心理上的优势重新赢回了西门庆的身体。在越来越亢奋和毫无节制的性爱中,西门庆走向灭亡,潘金莲在勾引陈敬济后迎来了自己的第二春,事发后被吴月娘走出家门。陈敬济回东京凑钱娶潘金莲,此时武松回到了清河县,他得知了武大郎的死亡真相,假称娶潘金莲过门,潘金莲大喜过望(纵观全书,她真正爱的人可能只有武松),在毫无防备的情况下被武松在洞房花烛夜剜心而亡,死后暴尸街头,多日后才回庞春梅为她下葬。

潘金莲的一生是悲惨的,无论是在张大户、武大郎家,还是跟着西门庆和陈敬济,她都没有“满足”过,更没有“幸福”过,这些东西似乎就在她不远的前方,她满心欢喜地跑过去,却一个都抓不住。她的风骚、泼辣和狠毒,是对生活中不如意的反抗,这种反抗为其他人带来了死亡和悲剧。甚至对自己的母亲,她也是嫌弃她,恶毒地诅咒她,咒骂她。可是潘姥姥冤枉吗?当初是潘姥姥一次次把潘金莲卖来卖去,只有西门庆才是潘金莲对自己婚姻的自主选择。前面说过,武松可能是她唯一真正爱过的人,但她却成了这个人的杀兄的仇人。从法律和正义的角度上来看,潘金莲的死一点都不冤,可是是“大快人心”之后,读者肯定会叹息一声。真的,潘金莲太可怜了,其他很多人尽管也受苦受难,但他们可能在即使一瞬间也幸福和快乐过,也有过甜美的回忆和烂漫的希望,但潘金莲没有,她的欢喜在于武松终于回来娶她了,这直接导致了她的死亡。



\section{李瓶儿}

\section{庞春梅}

\section{吴月娘}

\section{孟玉楼}

\section{孙雪娥}

\section{陈敬济}

\section{应伯爵}

\section{如意儿}
奶妈,娘家姓章,排行第四,32岁。亡夫叫熊旺儿。李瓶儿死后上位,与西门庆勾上。

\section{谢希大}

\section{花子繇}

\section{祝实念}

\section{孙天化}

\section{常峙节}

\section{白赉光}
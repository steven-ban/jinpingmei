《金瓶梅》刻画了几十个性格各异、形象立体、层次分明的人物,这些人物在作者兰陵笑笑生笔下如同有了生命,于字里行间活脱跳跃。

\section{西门庆}
西门庆是本书的主角,是众多角色的主线。他父母早丧,兄弟俱无,靠生药铺艰难创业发家,有了点钱就在县衙门里包揽诉讼,利用自己强大的智商、情商和运气,做成了清河县的千户,掌管了这里的刑事诉讼,积累了万贯家财。在这个过程里,他不仅和当地的商人、官员、大户交结并捆绑在一条船上,还通过翟千户的关系与中央的蔡京等人建立了利益关系,可谓上下通吃。值得注意的是,西门庆的家产刚开始并不算特别富有,李瓶儿把自己的财产送给了他,才让他事业真正红火起来。

在地方上,西门庆可谓飞扬跋扈。他利用自己的职权,贪赃枉法,在诉讼上可谓败坏到了家。对等自己的“上级”,他摇尾乞怜,从他们手里获得权力,并实心办事,讨好上级。对等同级,他尽力拉扰,“强强联合”,又是通婚又是参股,把自己的家产越做越大。对等下位者和平民百姓,他如同君主,丝毫不放在眼里。

作为一部描述“性”毫不顾忌的小说,西门庆与其他女人间的性事成了本书最引人注目的情节。西门庆对待性,如同对待权力一样,可谓毫不遮掩毫无保留。他几乎天天做夜夜做,不仅和自家的妾做,还常常勾引下人,甚至连男人也不放过。他如同饿狼般地勾引其他的良家妇女。而这些和他交欢的女人,不仅仅是看中了他的权力和财富,还看中了他有够也愿意给女人带来“快乐”。王婆所说的“潘驴邓小闲”,西门庆样样都占。吊诡的是,对于自己的正妻吴月娘,他却很少下手。他前妻死亡,只留下一个女儿,李瓶儿为他生了儿子但被潘金莲害死,只有等到死亡当天才由吴月娘生下儿子,但最终儿子出家,也没有继承他的家业。

西门庆对权力和性的追求,在我看来是一体的。如同那句著名的话,\emph{Everything in the world is about sex except sex. Sex is about power.}西门庆对性和权力这种赤裸裸的毫无羞耻的追求,源于他生命底色里的阳刚。《金瓶梅》里的男人,从贩夫走卒到高级官员,甚至是读圣贤书的人,无一不是在性上贪婪无厌,,但做事如此“坦荡”“自然”的,恐怕只有西门庆一个。

极具宿命论的是,正是这种对性的贪婪无厌害死了西门庆。胡僧药对于西门庆来说,是追求性快感的捷径和武器,也掏空了他的身体。西门庆死于和潘金莲的交欢,而后者是比他更疯狂的性瘾者,真是“一物降一物”。他一死,附着在他身上的如同蛆虫般男男女女成了反噬他的蚁群,他的家产也很快被人瓜分殆尽。

\section{潘金莲}

\section{李瓶儿}

\section{庞春梅}

\section{吴月娘}

\section{孟玉楼}

\section{孙雪娥}

\section{陈敬济}

\section{应伯爵}

\section{如意儿}
奶妈,娘家姓章,排行第四,32岁。亡夫叫熊旺儿。李瓶儿死后上位,与西门庆勾上。

\section{谢希大}

\section{花子繇}

\section{祝实念}

\section{孙天化}

\section{常峙节}

\section{白赉光}
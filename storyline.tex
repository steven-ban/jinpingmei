按照回目对《金瓶梅》的主要情节进行概括。这里采用的版本是万历本。



\begin{longtable}{|c|p{0.3\textwidth}|p{0.6\textwidth}|}
\caption{《金瓶梅》按回目主要情节}
\label{tab:storyline} \\
\hline
\heiti{回目} & \heiti{回名} & \heiti{主要事件} \\
\hline	
\endhead % 每页重复标题行

1 & 西门庆热结十弟兄,武二郎冷遇亲哥嫂 &  \\
2 & 俏潘娘帘下勾情,老王婆茶坊说技 &  \\
3 & 定挨光王婆受贿,设圈套浪子么挑 & \\
4 & 赴巫山潘氏幽欢,闹茶坊郓哥义愤 & \\
5 & 捉奸情郓哥定计,饮鸩药武大遭殃 & \\
6 & 何九受贿瞒天,王婆帮闲遇雨 & \\
7 & 薛媒婆说娶孟三儿,杨姑娘气骂张四舅 & \\
8 & 盼情郎佳人占鬼卦,烧夫灵和尚听淫声 & \\
9 & 西门庆偷娶潘金莲,武都头误打李皂隶 & \\
10 & 义士充配孟州道,妻妾玩赏芙蓉亭 & \\
11 & 潘金莲激打孙雪娥,西门庆梳笼李桂姐 & \\
12 & 潘金莲私仆受辱,刘理星魇胜求财 & \\
13 & 李瓶姐墙头密约,迎春儿隙底私窥 & \\
14 & 花子虚因气丧身,李瓶儿迎奸赴会 & \\
15 & 佳人笑赏玩灯楼,狎客帮嫖丽春院 & \\
16 & 西门庆择吉佳期,应伯爵追欢喜庆 & \\
17 & 宇给事劾倒杨提督,李瓶儿许嫁蒋竹山 & \\
18 & 赂相府西门脱祸,见娇娘敬济销魂 & \\
19 & 草里蛇逻打蒋竹山,李瓶儿情感西门庆 & \\
20 & 傻帮闲趋奉闹华筵,痴子弟争锋毁花院 & \\
21 & 吴月娘扫雪烹茶,应伯爵替花邀酒 & \\
22 & 蕙莲儿偷期蒙,爱春梅姐正色闲邪 & \\
23 & 赌棋枰瓶儿输钞,觑藏春潘氏潜踪 & \\
24 & 敬济元夜戏娇姿,惠祥怒詈来旺妇& \\
25 & 吴月娘春昼秋千,来旺儿醉中谤仙& \\
26 & 来旺儿递解徐州,宋蕙莲含羞自缢& \\
27 & 李瓶儿私语翡翠轩,潘金莲醉闹葡萄架& \\
28 & 陈敬济徼幸得金莲,西门庆糊涂打铁棍& \\
29 & 吴神仙冰鉴定终身,潘金莲兰汤邀午战& \\
30 & 蔡太师擅恩锡爵,西门庆生子加官& \\
31 & 琴童儿藏壶构衅,西门庆开宴为欢& \\
32 & 李桂姐趋炎认女,潘金莲怀妒惊儿& \\
33 & 陈敬济失钥罚唱,韩道国纵妇争锋& \\
34 & 献芳樽内室乞恩,受私贿后庭说事& \\
35 & 西门庆为男宠报仇,书童儿作女妆媚客& \\
36 & 翟管家寄书寻女子,蔡状元留饮借盘缠& \\
37 & 冯妈妈说嫁韩爱姐,西门庆包占王六儿& \\
38 & 王六儿棒槌打捣鬼,潘金莲雪夜弄琵琶& \\
39 & 寄法名官哥穿道服,散生日敬济拜冤家& \\
40 & 抱孩童瓶儿希宠,妆丫鬟金莲市爱& \\
41 & 两孩儿联姻共笑嬉,二佳人愤深同气苦& \\
42 & 逞豪华门前放烟火,赏元宵楼上醉花灯 & \\
43 & 争宠爱金莲惹气,卖富贵吴月攀亲 & 西门庆拿给李瓶儿四个金锭子,丢了一个。乔五太太拜访结亲 \\
44 & 避马房侍女偷金,下象棋佳人消夜 & 丢的金锭子是被二娘房里的丫鬟夏花儿拾的,被打了一顿 \\
45 & 应伯爵劝当铜锣,李瓶儿解衣银姐 & 几个人给西门庆送礼,其中包括一个大屏风 \\
46 & & \\
47 & & \\
48 & & \\
49 & & \\
50 & & \\
51 & & \\
52 & & \\
53 & & \\
54 & & \\
55 & & \\
56 & & \\
57 & & \\
58 & & \\
59 & & \\
60& & \\
61 & & \\
62 & & \\
63 & & \\
64 & & \\
65 & & \\
66 & & \\
67 & & \\
68 & & \\
69 & & \\
70 & & \\
71 & & \\
72 & & \\
73 & 潘金莲不愤忆吹箫,西门庆新试白绫带 & 潘金莲为西门庆做了取代银托子,长期给胡僧药。西门庆听《忆吹箫》回忆李瓶儿,潘金莲很生气。潘金莲打秋菊。 \\
74 & 潘金莲香腮偎玉,薛姑子佛口谈经 & 潘金莲与西门庆交欢。西门庆把李瓶儿的衣服送给潘金莲,给如意衣服。西门庆与官员会面。吴月娘听讲佛《黄氏宝卷》怀孕,导致之后孩子入佛门。 \\
75 & 因抱恙玉姐含酸,为护短金莲泼醋 & 西门庆与如意交欢,也尿到她口中。春梅骂走李桂姐。孟玉楼呕吐,西门庆喂她吃药,与她交欢。吴月娘与潘金莲互骂,难受,西门庆安慰她,次日请来任医官为她看病。 \\
76 & & \\
77 & & \\
78 & & \\
79 & 西门庆贪欲丧命,吴月娘失偶生儿 & 西门庆与王六儿交欢。西门庆与潘金莲交欢,潘金莲用了过多的胡僧药,加上西门庆这几天头晕虚弱,导致西门庆射精、射血、射冷气,阴囊肿胀,随后破裂,死亡。同时吴月娘生下儿子。 \\
80 & 潘金莲售色赴东床,李娇儿盗财归丽院 & 应伯爵等人出份子让人写吊文,但写吊文的人暗含讥讽。王六儿吊孝,吴月娘不接,后来听了吴大舅的劝才让人迎接。陈敬济与潘金莲偷情。李娇儿重回妓院,随后被张二官买作二房,应伯爵向张二官推荐潘金莲。 \\
81 & 韩道国拐财远遁,汤来保欺主背恩 & 韩道国从南方回来听说西门庆死了,拐走一千两银子,来保强占布铺欺负月娘。 \\
82 & 陈敬济弄一得双,潘金莲热心冷面 & 陈敬济与潘金莲偷情,被庞春梅撞见,庞春梅隐瞒这事并和陈敬济交合。潘金莲在陈敬济袖里有孟玉楼的簪子,对陈敬济生气。\\
83 & 秋菊含恨泄幽情,春梅寄柬谐佳会 & 秋菊发现陈敬济与潘金莲偷情,泄漏给下人和吴月娘,反被潘金莲知道,被潘金莲打了一顿。吴月娘为避免他人闲话隔开了两人,但两人克服困难依旧偷情。 \\
84 & 吴月娘大闹碧霞宫,曾静师化缘雪涧洞 & 吴月娘和吴大舅去岱岳庙进香,被道士欺骗,酒醉后当地太守妻弟欲强奸她,不成就逃跑,吴月娘和吴大舅也逃跑,遇见雪洞禅师,答应十五年后让孝哥做他徒弟。 \\
85 & 吴月娘识破奸情,春梅姐不垂别泪 & 潘金莲打掉与陈敬济的孩子,事情败露。如意举报,月娘撞见两人干事,让两人分开。陈敬济托薛嫂传信。月娘打发春梅离开。\\
86 & & \\
87 & & \\
88 & & \\
89 & & \\
90 & & \\
91 & & \\
92 & & \\
93 & & \\
94 & & \\
95 & & \\
96 & & \\
97 & & \\
98 & & \\
99 & & \\
100 & & \\


\hline
\end{longtable}
